\documentclass{article}
\usepackage{anysize}
\marginsize{7mm}{7mm}{5mm}{5mm}
\usepackage[ruled,vlined]{algorithm2e}
\LARGE\title{Movie Reviews}
\author{Varun Khare, GSSS Rao}
\date{March 2017 }
\begin{document}
\maketitle
\section{Golmaal}
The movie is fun to watch and leaves you laughing till the very end. We hope you also would enjoy the same amount of pleasure that we got in watching this.
\\
Light, smart, and downright funny, Golmaal is a story of clever people improvising their way through a thoroughly ridiculous series of deceptions.  It is a simply delightful movie, similar in feel to Hrishida's equally charming Chupke chupke. 
\\
With his freshly minted degree in accounting, Ram Prasad Sharma (Amol Palekar) needs a job.  To land a position with the stern businessman Bhavani Shankar (Utpal Dutt), Ram plays up his studious, serious side - dressing in traditional clothing, spouting aphorisms about the value of hard work, and concealing his passion for sports.  After Bhavani spots Ram skiving off work to watch a hockey match, the quick-thinking Ram invents a no-good, layabout twin, whom he calls (of course) Lakshman, and convinces Bhavani that the fellow he saw at the match was Lakshman, not Ram.  Bhavani, wanting to favor his protege's family, insists that "Lakshman" tutor Bhavani's daughter Urmila (Bindiya Goswami) in music.  Now Ram has to play two roles for Bhavani - the earnest respectful Ram, and the modern loudmouth, Lakshman.  Other improvisations force Ram to enlist an amateur actress, Kamla Shrivastav (Dina Pathak) to introduce to Bhavani as his mother.  Soon, when Kamla runs into Bhavani at a party, she too invents a twin for herself, so as not to blow Ram's cover.  It's not long before all hell breaks loose when Bhavani decides that the hard-working Ram is a perfect match for Urmila - she, of course, prefers the free-spirited Lakshman.
\\
It's hard to say what is most to love about Golmaal.  It's crisply paced and smartly scripted, full of the sort of linguistic humor that I especially love (and that was also in Chupke chupke) - Ram graces Bhavani with such upstanding shuddh Hindi that Bhavani can't quite keep up.  The physical humor is hilarious and masterfully performed.  Utpal Dutt pulls a diverse array of magnificent expressions as Bhavani - wistful as Ram praises the value of the mustache, horrified as Bhavani listens to Urmila recite shamefully modern lines from a play she is acting in.  Amol Palekar, too, is charming, adorable, and very, very funny as he squirms every which way to convince Bhavani that his two sides are two different people.  One cute running physical joke has Ram constantly tugging at his uncomfortably short kurta.  The kurta itself is the result of another of Hrishida's favorite devices, movie in-jokes.  Ram borrows the kurta from his childhood friend, the actor Deven Verma (playing himself), on a movie set - it's short because it's Asrani's kurta, settled on after rejecting Amitabh Bachchan's as too long, and Sanjeev Kumar's as too wide.
\\
Golmaal has too many funny set-pieces and running jokes to call them all out, but a few bear mention.  In one great sequence, Bhavani's widowed sister overhears Urmila rehearsing the melodramatic lines from her play, and concludes that Urmila is pregnant by a man who is now abandoning her.  Much hilarity ensues.  There are very funny running jokes about Ram's mustache.  Bhavani is nearly obsessed with the idea that the mustache makes the man, and Ram is equally attached to his own, which he sacrifices for the sake of his double life.  Every interaction between Ram and Bhavani is hilarious, whether showcasing Bhavani's over-the-top paternal affection for Ram Prasad, or his equally extreme disdain for Lakshman. 
\\
The greatest joy of Golmaal is that even in its comedy of errors, it resonates with everyday experience.  Ram is neither as earnest as Ram Prasad, nor as breezy as Lakshman - his true character is a blend of both.  But most of us find that different aspects of who we are emerge in our interactions with various people.  I do not show the same face to the head of my department at work as I do to my closest friend.  By amplifying this sort of common multiplicity and taking it to an absurd extreme, Hrishikesh Mukherjee makes Ram - and Golmaal - supremely relatable.  Add in a lot of laughs and a few catchy songs, and the sum is just a thoroughly delightful movie. 
\section{movie 2}
\end{document}
