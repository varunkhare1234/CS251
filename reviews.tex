\documentclass{article}
\usepackage{anysize}
\marginsize{7mm}{7mm}{5mm}{5mm}
\usepackage[ruled,vlined]{algorithm2e}
\LARGE\title{Movie Reviews}
\author{Varun Khare, GSSS Rao}
\date{March 2017 }
\begin{document}
\maketitle
\section{Sholay}
This is a really old movie from 1975. Its cast has many famous actors like Dharmendra, Hema Malini, Sanjeev Kumar, Amitabh Bachan etc.
\\
This movie has excellent dialogues, excellent music and an interesting story/plot. The story is about a former policeman(Sanjeev Kumar) whose family is murdered by a bandit. This incident makes the police officer enlist the services of two outlaws Veeru (Dharmendra) and Jai (Amitabh Bachchan) to capture the notorious dacoit Gabbar Singh (Amjad Khan).
\\
Inspite of being one of the most techinically correct movies I have noticed a few mistakes here and there, I have noticed just two see if you can find more;

1) When Amitabh and Dharmendra do a runaway from prison, they go to Surma Bhopali to collect their share of money (1000 bucks), Dhamendra says ``surma bhai hum apne hazar rupey le ne aaye hai" after a brief pause Surma bhopali says ``acha acha hazar rupey...........doo soo ke das hai gin lo" But 10x200=2000 bucks. I am not sure why he got paid extra?

2) When Amitabh ''dies'' in the lap of Dharmendra, initially hig legs appear to be crossed. But, after some time they appear wide and separate. Probably, an editing error.
\\
I would say that if you liked "Once upon a time in the west", "The magnificent seven" and "A fistful of dollars", you would definitely love "Sholay". It takes all the good ingredients of a western, spices it with a lot of Indian ingredients and lets it cook until you get one of the finest "curry" Bollowyood movie. 
It's a movie that you can watch many times, and hardcore Bollywood fans do. But even if you just watch Sholay once, you won't be disappointed. Almost everyone in India knows of this amazing piece of work. 
\\
If you have not watched this move, then make definitely make it your priority to watch it at least once in your life, as this is definitely one of the movies that every person needs to experience.
\section{Golmaal}
The movie is fun to watch and leaves you laughing till the very end. We hope you also would enjoy the same amount of pleasure that we got in watching this.
\\
Light, smart, and downright funny, Golmaal is a story of clever people improvising their way through a thoroughly ridiculous series of deceptions.  It is a simply delightful movie, similar in feel to Hrishida's equally charming Chupke chupke. 
\\
With his freshly minted degree in accounting, Ram Prasad Sharma (Amol Palekar) needs a job.  To land a position with the stern businessman Bhavani Shankar (Utpal Dutt), Ram plays up his studious, serious side - dressing in traditional clothing, spouting aphorisms about the value of hard work, and concealing his passion for sports.  After Bhavani spots Ram skiving off work to watch a hockey match, the quick-thinking Ram invents a no-good, layabout twin, whom he calls (of course) Lakshman, and convinces Bhavani that the fellow he saw at the match was Lakshman, not Ram.  Bhavani, wanting to favor his protege's family, insists that "Lakshman" tutor Bhavani's daughter Urmila (Bindiya Goswami) in music.  Now Ram has to play two roles for Bhavani - the earnest respectful Ram, and the modern loudmouth, Lakshman.  Other improvisations force Ram to enlist an amateur actress, Kamla Shrivastav (Dina Pathak) to introduce to Bhavani as his mother.  Soon, when Kamla runs into Bhavani at a party, she too invents a twin for herself, so as not to blow Ram's cover.  It's not long before all hell breaks loose when Bhavani decides that the hard-working Ram is a perfect match for Urmila - she, of course, prefers the free-spirited Lakshman.
\\
It's hard to say what is most to love about Golmaal.  It's crisply paced and smartly scripted, full of the sort of linguistic humor that I especially love (and that was also in Chupke chupke) - Ram graces Bhavani with such upstanding shuddh Hindi that Bhavani can't quite keep up.  The physical humor is hilarious and masterfully performed.  Utpal Dutt pulls a diverse array of magnificent expressions as Bhavani - wistful as Ram praises the value of the mustache, horrified as Bhavani listens to Urmila recite shamefully modern lines from a play she is acting in.  Amol Palekar, too, is charming, adorable, and very, very funny as he squirms every which way to convince Bhavani that his two sides are two different people.  One cute running physical joke has Ram constantly tugging at his uncomfortably short kurta.  The kurta itself is the result of another of Hrishida's favorite devices, movie in-jokes.  Ram borrows the kurta from his childhood friend, the actor Deven Verma (playing himself), on a movie set - it's short because it's Asrani's kurta, settled on after rejecting Amitabh Bachchan's as too long, and Sanjeev Kumar's as too wide.
\\
Golmaal has too many funny set-pieces and running jokes to call them all out, but a few bear mention.  In one great sequence, Bhavani's widowed sister overhears Urmila rehearsing the melodramatic lines from her play, and concludes that Urmila is pregnant by a man who is now abandoning her.  Much hilarity ensues.  There are very funny running jokes about Ram's mustache.  Bhavani is nearly obsessed with the idea that the mustache makes the man, and Ram is equally attached to his own, which he sacrifices for the sake of his double life.  Every interaction between Ram and Bhavani is hilarious, whether showcasing Bhavani's over-the-top paternal affection for Ram Prasad, or his equally extreme disdain for Lakshman. 
\\
The greatest joy of Golmaal is that even in its comedy of errors, it resonates with everyday experience.  Ram is neither as earnest as Ram Prasad, nor as breezy as Lakshman - his true character is a blend of both.  But most of us find that different aspects of who we are emerge in our interactions with various people.  I do not show the same face to the head of my department at work as I do to my closest friend.  By amplifying this sort of common multiplicity and taking it to an absurd extreme, Hrishikesh Mukherjee makes Ram - and Golmaal - supremely relatable.  Add in a lot of laughs and a few catchy songs, and the sum is just a thoroughly delightful movie. 


\section{Munna Bhai M.B.B.S(2003)}

\textbf{Cast: Sanjay Dutt,Sunil Dutt,Gracy Singh,Arshad Warsi, Boman Irani}
\textbf{Director: Rajkumar Hirani }  
\textbf{Producer: Vidhu Vinod Chopra }

\newline
\newline
\textbf{In a nutshell: } A bollywood movie with a tapori touch with well timed humour and emotions depiction along with tight story line.
\newline
Munna BHai M.B.B.S is one of the boolywood movies with well time humour in present days. The story line is simple, yet bounding. The movie is about the Murli Prasad Sharma(\textbf{Sanjay Dutt}), who presently lives in Mumbai and is the local "bhai". He is indulged in activities like kidnapping, obtaining money forcefully and beating. Somehow he is bit afraid of his father, a character played by \textbf{Sunil Dutt}. Munna's father always wanted to see his son become a doctor and presently thinks that, his son is a doctor in Mumbai. Munna never had the courage to tell his father about his daily work. Munna is always accompanied by his friend named "Circuit", whose character is played by \textbf{Arshad Warsi}. The story takes a fold, when Munna's father decides to come and see his son at Mumbai. Further it's shown how Munna deals with the issue, opens a false hospital and tackles several other problem. \newline
Overall the movie is light and is gripping at moments. All the characters did justice with their roles. Though the screen time for Arshad Warsi and Gracy Singh was bit less, but their performance was worth watching. All together, it is a nice movie to watch along with family and enjoy the weekened. One could expect it to work well at the box office.
\newline
\textbf{Rating: 4.5/5}


\section{Lage Raho Munna Bhai(2006)}

\textbf{Cast: Sanjay Dutt,Vidya Balan,Arshad Warsi, Boman Irani}
\textbf{Director: Rajkumar Hirani }  

\newline
\textbf{In a nutshell: }Not the sequel of Munna Bhai M.B.B.S. Independent story with the same touch of humour and emotions.
\newline
Lage Raho Munna Bhai is all together different film and is not the sequel of the movie released earlier. The name of the protagonist of the story is same as that of the earlier film, "Munna". The character of Munna's friend is also the same. The character of circuit is played by Arshad Warsi once again. It's whole together is a new adventure and is about the fight of Munna against a corrupt man following strictly the Gandhian Principle. The role of Munna's love is played by Vidya Balan and the character of the corrupt is played by Boman Irani. Initially in the movie, Munna tries to impress Janvi(Vidya Balan) just by showing that he follows the Gandhian Principle. Things chages drastically as the movie unfolds. 
\newline
The movie is humorous in nature and is mass appealing. Again it is a family movie along with several nice performances by the artists. The movie depicts it's theme brilliantly and is a must watch to feel some freshness. \newline
\newline
\textbf{Rating: 4.0/5}
\newline



\end{document}
