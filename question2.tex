\documentclass{article}
\usepackage{anysize}
\marginsize{7mm}{7mm}{5mm}{5mm}
\usepackage[ruled,vlined]{algorithm2e}
\LARGE\title{Merge Sort}
\author{Varun Khare}
\date{Feburary 2017 }
\begin{document}
\maketitle
\section{Description}
Merge Sort is the most efficient algorithm to sort a given array of numbers. It is based on divide and conquer strategy. Its time complexity is O(nlogn). 

Intuitively, it operates as follows:
\begin{itemize}

	\item Divide: Divide the n-element sequence to be sorted into two subsequences of n/2 elements each.
	\item Conquer: Sort the two subsequences recursively using merge sort.
	\item Combine: Merge the two sorted subsequences to produce the sorted answer.
\end{itemize}    

		The recursion stops at a sequence of length of 1 as it is already a sorted array.
\subsection{Pseudo Code}

		For it's implementation we use two functions:
		\begin{itemize}
			\item MERGE -- To merge the two already sorted halves of the array.
			\item MERGE\_SORT -- The main algorithm which recursively sorts the array. 
		\end{itemize}	

		Pseudo code for MERGE\\
		\begin{algorithm}[H]
			\caption{MERGE(A,p,q,r)}
			\SetAlgoLined 
			$n_1 \gets q-p+1$ \\ 
			$n_2 \gets r-q$ \\
			Create arrays $L[1 .... n_1 + 1] $ and $ R[1 .... n_2 + 1]$ \\
			\For{\textit{$i \gets 1$ TO $n_1$ }} {
				$L[i] \gets A[p + i -1]$ 
			}
			\For{$j\leftarrow 1$ TO $n_2$}{
				$R[j] \gets A[q + j]$
			}
			$L[n_1 + 1] \gets \infty$ \\
			$R[n_1 + 1] \gets \infty$ \\
			$i \gets 1$ \\
			$j \gets 1$ \\
			\For{\textit{$k \gets p$ TO $r$}}{
				\If {$R[j] \ge L[i]$}{
					$A[k] \gets L[i]$ \\
			$i \gets i + 1$}
			\Else{
				$A[k] \gets R[j]$ \\
			     $j \gets j+1$
			} 
			}
		\end{algorithm}
        \newpage
		

		Pseudo code for MERGE\_SORT \\\\	     
			     \begin{algorithm}[H]
				     \caption{MERGE\_SORT(A,p,r)}
				     \If {$r \ge p + 1$}{
					     $q \gets \lfloor (p+1)/2 \rfloor$ \\
				     $MERGE\_SORT(A, p, q)$ \\
				     $MERGE\_SORT(A, q+1, r)$ \\
				     $MERGE(A, p, q ,r)$
				     }
			     \end{algorithm}
\end{document}
